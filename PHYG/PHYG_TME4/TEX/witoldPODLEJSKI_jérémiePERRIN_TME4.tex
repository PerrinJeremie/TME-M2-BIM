\documentclass[]{article}

\usepackage{float}
\usepackage[utf8]{inputenc}
\usepackage[T1]{fontenc}
\usepackage[english]{babel} % If you write in English
\usepackage{graphicx}
\usepackage{amsthm}
\usepackage{amssymb,amsmath}
\usepackage{tabularx}
\usepackage[dvipsnames]{xcolor}

\usepackage{fancyvrb}

\newtheorem{theorem}{Theorem}[section]
\newtheorem{lemma}[theorem]{Lemma}

\theoremstyle{definition}
\newtheorem{definition}{Definition}[section]

% redefine \VerbatimInput
\RecustomVerbatimCommand{\VerbatimInput}{VerbatimInput}%
{fontsize=\footnotesize,
	%
	frame=lines,  % top and bottom rule only
	framesep=1em, % separation between frame and text
	rulecolor=\color{Gray},
	%,
	labelposition=topline
}

%opening
\title{TME4 - PHYG}
\author{Witold PODLEJSKI \& Jérémie PERRIN}

\begin{document}

\maketitle

\section{Exercise 1}
\subsection{Question 1}
\subsection{Question 2}
\section{Exercise 2}

\begin{figure}[H]
	\includegraphics*[width = \linewidth]{image/Nj_ex2.pdf}
	\caption{ The tree inferred with the Neighbor-Joining algoritm on one gene familly }
\end{figure}

\begin{figure}[H]
	\includegraphics*[width = \linewidth]{image/Ml_ex2.pdf}
	\caption{ The tree inferred with the Most-Likelihood algoritm on one gene familly}
\end{figure}

\section{Exercise 3}

\begin{figure}[H]
	\includegraphics*[width = \linewidth]{image/Nj_ex3.pdf}
	\caption{ The tree inferred with the Neighbor-Joining algoritm on all gene famillies }
\end{figure}

\begin{figure}[H]
	\includegraphics*[width = \linewidth]{image/Ml_ex3.pdf}
	\caption{ The tree inferred with the Most-Likelihood algoritm on all gene famillies}
\end{figure}

\section{Exercise 4}

\begin{figure}[H]
	\includegraphics*[width = \linewidth]{image/NJ_supertree.pdf}
	\caption{ The tree inferred with the Neighbor-Joining algoritm and a supertree combination algoritm on all gene famillies }
\end{figure}

\begin{figure}[H]
	\includegraphics*[width = \linewidth]{image/ML_supertree.pdf}
	\caption{ The tree inferred with the Most-Likelihood algoritm and a supertree combination algoritm on all gene famillies}
\end{figure}

\section{Exercise 5}

\end{document}