\documentclass[]{article}

\usepackage{float}
\usepackage[utf8]{inputenc}
\usepackage[T1]{fontenc}
\usepackage[english]{babel} % If you write in English
\usepackage{graphicx}
\usepackage{amsthm}
\usepackage{amssymb,amsmath}

\usepackage[dvipsnames]{xcolor}

\usepackage{fancyvrb}

\newtheorem{theorem}{Theorem}[section]
\newtheorem{lemma}[theorem]{Lemma}

\theoremstyle{definition}
\newtheorem{definition}{Definition}[section]

% redefine \VerbatimInput
\RecustomVerbatimCommand{\VerbatimInput}{VerbatimInput}%
{fontsize=\footnotesize,
	%
	frame=lines,  % top and bottom rule only
	framesep=1em, % separation between frame and text
	rulecolor=\color{Gray},
	%,
	labelposition=topline
}

%opening
\title{TME2 - PHYG}
\author{Witold PODLEJSKI \& Jérémie PERRIN}

\begin{document}

\maketitle

\section{Exercise 1 : Genome evolution simulation}
\subsection{Question 1}

 
\subsection{Question 2}


\subsection{Question 3}

\section{Exercise 2 : Implementation of simple events}
Cf. folder "code".

\section{Exercise 3 : Reconstruction using characters}
\subsection{Question 1}

Cf. folder "code/results/[1-10]".

\subsection{Question 2}

\begin{tabular}{|*{11}{c|}}
	\hline
	& Sim 1  & Sim 2  & Sim 3  & Sim 4 & Sim 5 & Sim 6 & Sim 7 & Sim 8 & Sim 9 & Sim 10 \\
	\hline
	Inversion  & 7  & 10  & 11  & 11 & 12 & 10 & 7 & 8 &7&10 \\
	\hline
	Translocation &8&1&5&1&4&7&6&2&1&6 \\
	\hline
	Duplication  & 0&1&0&1&2&0&1&3&3&0 \\
	\hline
	Deletions  & 0&2&1&1&2&0&3&2&4&0 \\
	\hline
	Fusions   &0&0&0&0&0&0&0&0&0&0 \\
	\hline
	Fissions &0&0&0&0&0&0&0&0&0&0 \\
	\hline
	WGD  &0&0&0&0&0&0&0&0&0&0 \\
	\hline
\end{tabular}



\subsection{Question 3}



\subsection{Question 4}

\begin{figure}[h!]
	\includegraphics*[width = \linewidth]{image/tree_1_true.pdf}
	\caption{\label{nj2} Parsimony tree without enlarged malleolus }
\end{figure}

\end{document}

