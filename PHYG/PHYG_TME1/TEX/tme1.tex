\documentclass[]{article}

\usepackage[utf8]{inputenc}
\usepackage[T1]{fontenc}
\usepackage[english]{babel} % If you write in English
\usepackage{a4wide}
\usepackage{graphicx}

\usepackage[dvipsnames]{xcolor}

\usepackage{fancyvrb}

% redefine \VerbatimInput
\RecustomVerbatimCommand{\VerbatimInput}{VerbatimInput}%
{fontsize=\footnotesize,
	%
	frame=lines,  % top and bottom rule only
	framesep=1em, % separation between frame and text
	rulecolor=\color{Gray},
	%,
	labelposition=topline
}

%opening
\title{TME1 - PHYG}
\author{Jérémie Perrin}

\begin{document}

\maketitle

\section{Exercise 1 : UPGMA}
\begin{figure}[h]
	\includegraphics*[width = \linewidth, height =0.7\textheight]{ex1.jpg}
\end{figure}

\section{Exercise 2 : NJ}
\begin{figure}[h]
	\includegraphics*[width = \linewidth, height =0.7\textheight]{ex2.jpg}
\end{figure}
The most reliable tree is the one from Neighbor Joining. The matrix being additive, the Neighbor Joining algorithm constructs a tree which conserves the distance. Since the matrix is not ultrametric the UPGMA algorithm does not build a tree which conserves the distances. UPGMA constructs a rooted tree with all leaves being at same distance from the root, a property not all trees have.

\newpage

\section{Exercise 3 : PAH}
\subsection{Question 1}
\VerbatimInput[	label=\fbox{\color{Black}Aligned sequences - Clustal}]{3_1.txt}

\subsection{Question 2}
\VerbatimInput[	label=\fbox{\color{Black}Distance Matrix - Protdist}]{3_2.txt}

\subsection{Question 3}
\begin{figure}[h]
\includegraphics*[width = \linewidth]{NJ_PH4H.pdf}
\caption{NJ tree based on PH4H}
\end{figure}
\begin{figure}[h]
	\includegraphics*[width = \linewidth]{UPGMA_PH4H.pdf}
	\caption{UPGMA tree based on PH4H}
\end{figure}

\subsection{Question 4}
The trees are the same if you consider that the root is actually the leaf. The topology is the same and distance very alike.


\section{Exercise 4 : CFTR}
\subsection{Question 1}
\VerbatimInput[	label=\fbox{\color{Black}Distance Matrix - Protdist}]{4_1.txt}

\newpage

\subsection{Question 2}
\begin{figure}[h]
	\includegraphics*[width = 0.9\linewidth]{NJ_CFTR.pdf}
	\caption{NJ tree based on CFTR}
\end{figure}

\begin{figure}[!h]
	\includegraphics*[width = \linewidth]{UPGMA_CFTR.pdf}
	\caption{UPGMA tree based on CFTR}
\end{figure}

\newpage 

\subsection{Question 3}
In UPGMA : Distant speciation and separate from the rest.
IN NJ  : Also look separate from the rest.

In the distance matrix the distance value is 0.098185. CFTR is very similar in mouse and rat. And both species are very different from all other species.

\newpage 

\subsection{Question 4}
\VerbatimInput[	label=\fbox{\color{Black}Simplified Trees}]{4_4.txt}

The pig is closer to bovine. The neighbor joining tree is correct, while UPGMA puts the horse closer to the pig.

\newpage

\section{Exercise 5 : P53}
\subsection{Question 1}
\begin{figure}[!h]
	\includegraphics*[width = \linewidth, height = 0.85\textheight]{NJ_P53.pdf}
	\caption{NJ tree based on P53}
\end{figure}

\subsection{Question 2}
\VerbatimInput[	label=\fbox{\color{Black}Simplified Trees}]{5_2.txt}

\subsection{Question 3}
Based on the simplified tree, we observe that P53 NJ tree is correct while the one based on CFTR is not.
\end{document}


