\documentclass[]{article}

\usepackage[utf8]{inputenc}
\usepackage[T1]{fontenc}
\usepackage[english]{babel} % If you write in English
\usepackage{a4wide}
\usepackage{graphicx}
\usepackage{amsthm}
\usepackage{amssymb,amsmath}

\usepackage[dvipsnames]{xcolor}

\usepackage{fancyvrb}

% redefine \VerbatimInput
\RecustomVerbatimCommand{\VerbatimInput}{VerbatimInput}%
{fontsize=\footnotesize,
	%
	frame=lines,  % top and bottom rule only
	framesep=1em, % separation between frame and text
	rulecolor=\color{Gray},
	%,
	labelposition=topline
}

%opening
\title{TME2 - PHYG}
\author{Witold PODLEJSKI \& Jérémie PERRIN}

\begin{document}

\maketitle

\section{Exercise 1 : Parsimony}
\subsection{Question 1}
\subsection{Question 2}
\subsection{Question 3}
\subsection{Question 4}
\subsection{Question 5}

%\begin{figure}[h]
%	\includegraphics*[width = \linewidth, height =0.7\textheight]{ex1.jpg}
%\end{figure}

\section{Exercise 2 : Reconstruction using reversal distances}
\subsection{Question 1}
\subsection{Question 2}
\subsection{Question 3}
\subsection{Question 4}
\subsection{Question 5}
\subsection{Question 6}
%\subsection{Question 1}
%\VerbatimInput[	label=\fbox{\color{Black}Aligned sequences - Clustal}]{3_1.txt}
\section{Exercise 3 : Reconstruction using characters}
\subsection{Question 1}
\subsection{Question 2}
\subsection{Question 3}
\subsection{Question 4}

\end{document}


